% Options for packages loaded elsewhere
% Options for packages loaded elsewhere
\PassOptionsToPackage{unicode}{hyperref}
\PassOptionsToPackage{hyphens}{url}
\PassOptionsToPackage{dvipsnames,svgnames,x11names}{xcolor}
%
\documentclass[
  letterpaper,
  DIV=11,
  numbers=noendperiod]{scrartcl}
\usepackage{xcolor}
\usepackage{amsmath,amssymb}
\setcounter{secnumdepth}{-\maxdimen} % remove section numbering
\usepackage{iftex}
\ifPDFTeX
  \usepackage[T1]{fontenc}
  \usepackage[utf8]{inputenc}
  \usepackage{textcomp} % provide euro and other symbols
\else % if luatex or xetex
  \usepackage{unicode-math} % this also loads fontspec
  \defaultfontfeatures{Scale=MatchLowercase}
  \defaultfontfeatures[\rmfamily]{Ligatures=TeX,Scale=1}
\fi
\usepackage{lmodern}
\ifPDFTeX\else
  % xetex/luatex font selection
\fi
% Use upquote if available, for straight quotes in verbatim environments
\IfFileExists{upquote.sty}{\usepackage{upquote}}{}
\IfFileExists{microtype.sty}{% use microtype if available
  \usepackage[]{microtype}
  \UseMicrotypeSet[protrusion]{basicmath} % disable protrusion for tt fonts
}{}
\makeatletter
\@ifundefined{KOMAClassName}{% if non-KOMA class
  \IfFileExists{parskip.sty}{%
    \usepackage{parskip}
  }{% else
    \setlength{\parindent}{0pt}
    \setlength{\parskip}{6pt plus 2pt minus 1pt}}
}{% if KOMA class
  \KOMAoptions{parskip=half}}
\makeatother
% Make \paragraph and \subparagraph free-standing
\makeatletter
\ifx\paragraph\undefined\else
  \let\oldparagraph\paragraph
  \renewcommand{\paragraph}{
    \@ifstar
      \xxxParagraphStar
      \xxxParagraphNoStar
  }
  \newcommand{\xxxParagraphStar}[1]{\oldparagraph*{#1}\mbox{}}
  \newcommand{\xxxParagraphNoStar}[1]{\oldparagraph{#1}\mbox{}}
\fi
\ifx\subparagraph\undefined\else
  \let\oldsubparagraph\subparagraph
  \renewcommand{\subparagraph}{
    \@ifstar
      \xxxSubParagraphStar
      \xxxSubParagraphNoStar
  }
  \newcommand{\xxxSubParagraphStar}[1]{\oldsubparagraph*{#1}\mbox{}}
  \newcommand{\xxxSubParagraphNoStar}[1]{\oldsubparagraph{#1}\mbox{}}
\fi
\makeatother


\usepackage{longtable,booktabs,array}
\usepackage{calc} % for calculating minipage widths
% Correct order of tables after \paragraph or \subparagraph
\usepackage{etoolbox}
\makeatletter
\patchcmd\longtable{\par}{\if@noskipsec\mbox{}\fi\par}{}{}
\makeatother
% Allow footnotes in longtable head/foot
\IfFileExists{footnotehyper.sty}{\usepackage{footnotehyper}}{\usepackage{footnote}}
\makesavenoteenv{longtable}
\usepackage{graphicx}
\makeatletter
\newsavebox\pandoc@box
\newcommand*\pandocbounded[1]{% scales image to fit in text height/width
  \sbox\pandoc@box{#1}%
  \Gscale@div\@tempa{\textheight}{\dimexpr\ht\pandoc@box+\dp\pandoc@box\relax}%
  \Gscale@div\@tempb{\linewidth}{\wd\pandoc@box}%
  \ifdim\@tempb\p@<\@tempa\p@\let\@tempa\@tempb\fi% select the smaller of both
  \ifdim\@tempa\p@<\p@\scalebox{\@tempa}{\usebox\pandoc@box}%
  \else\usebox{\pandoc@box}%
  \fi%
}
% Set default figure placement to htbp
\def\fps@figure{htbp}
\makeatother





\setlength{\emergencystretch}{3em} % prevent overfull lines

\providecommand{\tightlist}{%
  \setlength{\itemsep}{0pt}\setlength{\parskip}{0pt}}



 


\KOMAoption{captions}{tableheading}
\makeatletter
\@ifpackageloaded{caption}{}{\usepackage{caption}}
\AtBeginDocument{%
\ifdefined\contentsname
  \renewcommand*\contentsname{Table of contents}
\else
  \newcommand\contentsname{Table of contents}
\fi
\ifdefined\listfigurename
  \renewcommand*\listfigurename{List of Figures}
\else
  \newcommand\listfigurename{List of Figures}
\fi
\ifdefined\listtablename
  \renewcommand*\listtablename{List of Tables}
\else
  \newcommand\listtablename{List of Tables}
\fi
\ifdefined\figurename
  \renewcommand*\figurename{Figure}
\else
  \newcommand\figurename{Figure}
\fi
\ifdefined\tablename
  \renewcommand*\tablename{Table}
\else
  \newcommand\tablename{Table}
\fi
}
\@ifpackageloaded{float}{}{\usepackage{float}}
\floatstyle{ruled}
\@ifundefined{c@chapter}{\newfloat{codelisting}{h}{lop}}{\newfloat{codelisting}{h}{lop}[chapter]}
\floatname{codelisting}{Listing}
\newcommand*\listoflistings{\listof{codelisting}{List of Listings}}
\makeatother
\makeatletter
\makeatother
\makeatletter
\@ifpackageloaded{caption}{}{\usepackage{caption}}
\@ifpackageloaded{subcaption}{}{\usepackage{subcaption}}
\makeatother
\usepackage{bookmark}
\IfFileExists{xurl.sty}{\usepackage{xurl}}{} % add URL line breaks if available
\urlstyle{same}
\hypersetup{
  pdftitle={77 Links for Student Support \& Experience},
  colorlinks=true,
  linkcolor={blue},
  filecolor={Maroon},
  citecolor={Blue},
  urlcolor={Blue},
  pdfcreator={LaTeX via pandoc}}


\title{77 Links for Student Support \& Experience}
\usepackage{etoolbox}
\makeatletter
\providecommand{\subtitle}[1]{% add subtitle to \maketitle
  \apptocmd{\@title}{\par {\large #1 \par}}{}{}
}
\makeatother
\subtitle{Department of Economics}
\author{}
\date{}
\begin{document}
\maketitle


\phantomsection\label{quarto-tiles-index-static}
Assessment \& Timetable Information

{}

Timetable Information

{}

SEAtS Mobile app

{}

Examination Timetable

{}

Extensions

{}

eVision for Extenuating Circumstances

{}

Academic Misconduct

{}

Student Appeals

{}

Student Union Advise \& Support

{}

Progression and Awards

Key Economics Staff for Student Support

{}

Personal Tutor Year 1

{}

Personal Tutor Year 2

{}

Personal Tutor Year 3

{}

UG Programmes Director

{}

Personal Tutor \& PG Director

{}

Course Officers UG

{}

Course Officers PG

{}

Student Experience Director

{}

Head of Education

{}

SPGA Student Wellbeing Support

{}

Support@City

{}

Student Health \& Wellbeing e-Referral

{}

Student Health \& Wellbeing

{}

PG online Authorised Absence form

{}

Authorised Absence for Research students via Research Manager

{}

Attendance and Engagement

Student Support services

{}

Library

{}

Study Spaces

{}

service now

{}

Student Guides to Educational Technology

{}

CityBuddy

{}

Apply to get a mentor

{}

Support@City

{}

Accommodation

{}

Uni Cares

{}

International Student \& Visa Advice

{}

Student Funding

{}

Student Health \& Wellbeing

{}

School Student Welfare teams

{}

Student Support Hub Advisers

{}

The Student Support Hub

{}

Academic English support

{}

Academic Skills

{}

Careers \& Employability

{}

Digital Skills

{}

AppsAnywhere

{}

International Student \& Visa Advice

{}

IT Service Desk

{}

Togetherall

{}

The Report+Support platform

{}

Stay City Safe

{}

Your guide to staying safe in London

Students Union

{}

Student's Union

{}

Student Societies

{}

Start a new society

{}

SU Events

{}

Sports

{}

Short Term Loans

{}

Programme rep training

{}

GetHeard@City

{}

Programme Rep nomination form

{}

Questions about Reps

Careers and Employability

{}

Employability Economics

{}

Careers Hub

{}

Careers Fairs and employer events

{}

Micro-Placements

{}

Professional Mentoring Scheme

{}

Unitemps

{}

Student Ambassador scheme

{}

Book a Career appointment

Other useful information

{}

Term Dates

{}

Registration

{}

Registration team

{}

Fees and Finance

{}

Evision-Get Official Letters and Pay Fees

{}

CitySport

{}

Faith, belief and culture at City St George's

{}

Health information

{}

Help and support \textbar{} Student Hub \textbar{} City, University of
London

Resources for staff

{}

Toolkit for Supporting Students

{}

personal tutors

{}

Wellbeing \& Mental Health E-Learning

{}

Staff Supporting Students

{}

Managers Supporting Staff

{}

Wellbeing Strategies

{}

Risks and Crises

{}

Attendance and Engagement Monitoring

{}

SEAtS

{}

find out more about SEAtS attendance and monitoring training here

{}

including Powerpoint slides to incorporate into Induction activities, as
well as more information on programme content on the Learning and
Teaching Hub

\section{Assessment \& Timetable
Information}\label{assessment-timetable-information}

\href{https://mytimetable.city.ac.uk/}{Timetable Information}

\href{https://studenthub.citystgeorges.ac.uk/academic-resources/learning-and-study-support/study-skills-and-support/attendance-and-engagement/count-me-in}{SEAtS
Mobile app}: view your timetable and keep track of your attendance

\href{https://studenthub.city.ac.uk/timetabling-term-dates-exams/exams/exam-timetable}{Examination
Timetable}: assessment dates will be also available within your Moodle
Module pages, and exam dates can also be on this link.

\href{https://cityuni.sharepoint.com/teams/CUoL-SASSStudentSupport/SitePages/SASS-Extensions.aspx}{Extensions}:
can be used when you need up to 1 week extra to complete coursework.

\href{https://cityuni.sharepoint.com/teams/CUoL-SASSStudentSupport/SitePages/SASS-Extensions.aspx}{Extenuating
Circumstances}: can be used when you cannot take an examination or
assessment by the time of the deadline. This means you apply for a
deferment of the assessment to the next available assessment period
(normally, that would be the resit period in summer).

\href{https://evision.city.ac.uk/}{eVision for Extenuating
Circumstances}: Apply for Extenuating Circumstances via no eVision no
later than 7 days after your assessment deadline.

Late submission policy for coursework: Any work that is submitted more
than 1 minute past the deadline but within 48 hours will have 10 marks
deducted as a penalty. Work submitted after 48 hours will receive a
zero.

\href{https://studenthub.city.ac.uk/help-and-support/academic-integrity-and-misconduct}{Academic
Misconduct}: any action that produces or seeks to produce an improper
advantage for you in relation to your assessment(s) or deliberately and
unnecessarily disadvantages other students. It includes, but it is not
limited to plagiarism, collusion and contract cheating.

\href{https://studenthub.city.ac.uk/help-and-support/extenuating-circumstances-complaints-appeals}{Student
Appeals}: The appeals process is intended for the very rare cases where
no resolution or remedy was possible at an earlier stage (i.e.~prior to
the Assessment Board approving the results of taught students or a
decision being made regarding the registration status or examination
results of research students).

\href{https://csgsu.co.uk/advice-and-support}{Student Union Advise \&
Support}: you can contact the Student Union for advise and support,
contact the union to make an appointment.

\href{https://studenthub.city.ac.uk/timetabling-term-dates-exams/progression-and-awards}{Progression
and Awards}: important information about your assessment results, your
programme and the support available to you and details of the timescales
if you want to make an appeal.

\section{Key Economics Staff for Student
Support}\label{key-economics-staff-for-student-support}

\textbf{Your Module Leader}\,is the first point of contact for any
queries about the content or organisation of the module you are
studying. you can also book a meeting with a Module Leader during their
designated `office hours'. Module leader details are available in
Moodle.

\textbf{Your Personal Tutor}\,acts as a mentor and guide you to any
support you may need. You can talk to your personal tutor about any
struggles with your studies and personal life.\,Your personal tutor also
can refer you to the University's more specialised services for
academic, mental health and wellbeing support.\,You can discuss your
career aspirations with your personal tutor and receive advice on how to
plan your studies accordingly.

\href{mailto:econ-y1-tutor@city.ac.uk}{Personal Tutor Year 1}: Dr
Seefat-E-Rabbi Khan

\href{mailto:econ-y2-tutor@city.ac.uk}{Personal Tutor Year 2}: P Klaus
Zauner

\href{mailto:econ-y3-tutor@city.ac.uk}{Personal Tutor Year 3}: P Keith
Pilbeam

\href{mailto:pipat.wongsa-art@city.ac.uk}{UG Programmes Director}: Dr
Wongsa Pipat

\href{mailto:lena.hassani-nezhad@city.ac.uk}{Personal Tutor \& PG
Director}: Dr lena.hassani-nezhad@city.ac.uk

\href{mailto:economics.ug@city.ac.uk}{Course Officers UG} and
\href{mailto:pgspga@city.ac.uk}{Course Officers PG}; or visit the Course
Office information desk in Room A129, College Building (Monday--Friday,
10:00 AM--5:00 PM).Course Officers handle the course and official
records of students, including:

\begin{itemize}
\tightlist
\item
  Moodle page enrolment
\item
  Programme or module changes
\item
  timetabling clashes
\item
  how to submit an assessment
\item
  grades releases.
\item
  advice on extensions
\item
  submitting an Extenuating circumstance.
\item
  Suspension and interruption of studies
\end{itemize}

\href{mailto:camilo.calderon@city.ac.uk}{Student Experience Director}:
Dr Camilo Calderon. The Student Experience Director has an overview of
all operational issues concerning your experience at the University.
They also act as a second point of contact in case you cannot reach to
your Personal Tutor.

\href{mailto:panagiotis.giannarakis@city.ac.uk}{Head of Education}: Dr
Panagiotis Giannarakis

\href{https://evision.city.ac.uk/}{eVision for Personal Tutor contact}:
How to contact: Details of student Personal Tutors will be available on
Evision.

\textbf{Student Welfare and Engagement Officers}\,are your first points
of contact for any wellbeing or student experience related issues. Some
topics which students discuss with Student Engagement Officers can
include, but are not limited to; stress and anxiety, general wellbeing
issues, financial stress, challenging living environments,
time-consuming work or caring commitments, and disabilities. Based on
the topic either a Student Welfare or Student Engagement Officer will
reply to your query and provide advice on how you can best access the
support services available to you at the University. They may also
suggest that you meet with them so they can get more details on the
support you need.

\href{https://outlook.office365.com/book/SchoolofPolicyandGlobalAffairsStudentWellbeingSupport@cityuni.onmicrosoft.com/?ae=true&ismsaljsauthenabled=true}{SPGA
Student Wellbeing Support}

\href{https://support.city.ac.uk/create-case/}{Support@City}: Log a case
here for support.

\href{https://support.city.ac.uk/eReferral/?_gl=1\%2A1yf5nzx\%2A_gcl_au\%2AMTUzMzI5MTY4Ny4xNzQ4NjAxODEy\%2A_ga\%2ANzcwMjIyODIzLjE3NDA2NTA3Mjg.\%2A_ga_YLSK0292X4\%2AczE3NTU1OTIyOTkkbzM2MiRnMSR0MTc1NTU5NDg4MCRqMzYkbDAkaDA.}{Student
Health \& Wellbeing e-Referral}: If you are experiencing mental health
Issues or have a diagnosed mental health condition or disability it is
important that you register with the Student Mental Health Services,
even if you are currently receiving support externally. The Student
Mental Health Service can put in place an individualized Student Support
Plan with reasonable adjustments recommendations. You can complete an
e-Referral to the Student Health \& Wellbeing service.

\href{https://outlook.office365.com/book/StudentHealthWellbeing@cityuni.onmicrosoft.com/s/fmv9wdkSi0m08PSZBNQghg2?ismsaljsauthenabled=true}{Student
Health \& Wellbeing}: if you would prefer to meet with a Student Health
and Wellbeing Engagement Advisor.

If you are unable to attend classes for more than seven calendar days
due to extenuating circumstances, you may request an Authorised Absence.
If approved, the period covered will be formally excluded from your
attendance calculations. Absences of seven days or fewer do not require
formal approval. How to apply:

\href{https://studenthub.citystgeorges.ac.uk/academic-resources/learning-and-study-support/study-skills-and-support/attendance-and-engagement/count-me-in}{UG
Authorised Absence form via SEAtS Mobile App}.

\href{https://cityunilondon.eu.qualtrics.com/jfe/form/SV_5hzQmuXfZBM5plk}{PG
online Authorised Absence form}

\href{https://researchmanager.city.ac.uk/do/city-login/login}{Authorised
Absence for Research students via Research Manager}.

\href{https://studenthub.citystgeorges.ac.uk/academic-resources/learning-and-study-support/study-skills-and-support/attendance-and-engagement}{Attendance
and Engagement}: Further guidance on managing absences, including
eligibility and evidence requirements, is available on the Student Hub
page.

\section{Student Support services}\label{student-support-services}

\href{https://libraryservices.city.ac.uk/}{Library}: provides printed
resources, study spaces (for group study, quiet study and individual
silent study), networked PCs and wi-fi across five levels.

\href{https://studenthub.city.ac.uk/information-technology/citynav}{Study
Spaces}: these spaces include quiet areas for focused work, silent study
zones, and more social spaces for collaborative learning.

\href{https://studenthub.city.ac.uk/information-technology/citynav}{CityNav}:
hor more information on where to find these Study Spaces.

\subsection{IT support}\label{it-support}

\href{https://www.city.ac.uk/itservicedesk}{service now}: If students
experience IT issues, a ticket can be logged on this link.

Alternatively, students can call the IT help desk on 0207 040 8181 or
visit the IT help desk in person where they can help with issues such as
wi-fi access etc. The helpdesk in located in University Building.

\href{https://city-uk-ett.libguides.com/student}{Student Guides to
Educational Technology}: guidance on the use of educational technology
at City St Georges, such as Moodle, Microsoft teams, etc., is available
here.

\href{https://buddy.city.ac.uk/apply/mentee/}{CityBuddy} is a second- or
third-year student who will mentor a first-year student on their course.
They are there to show new students all that City has to offer; answer
any questions they may have and provide a valuable insight into studying
on the course.

\href{https://cityunilondon.eu.qualtrics.com/jfe/form/SV_9M41s2FmiwmYVM2}{Apply
to get a mentor}

\href{https://support.city.ac.uk/create-case/?_gl=1\%2asv17k1\%2a_ga\%2aODU0NDA5NDU1LjE2MjA5ODEyNjI.\%2a_ga_YLSK0292X4\%2aMTY5NDcwNDk3Ni4xOTcuMS4xNjk0NzA1MDMwLjYuMC4w}{Support@City}:
it's the online system allowing students to submit and track their
queries through a dedicated portal. The platform brings together many of
City St George's student support services in a secure portal to better
understand and address students' needs.

The system includes the following teams:

\href{https://studenthub.city.ac.uk/help-and-support/accommodation-student-hub}{Accommodation}

\href{https://studenthub.city.ac.uk/help-and-support/city-cares}{Uni
Cares}: City's dedicated support network for care-experienced students,
young adult carers, young estranged students, refugees and asylum
seekers.

\href{https://studenthub.city.ac.uk/help-and-support/visa-advice-for-international-students}{International
Student \& Visa Advice}

\href{https://studenthub.city.ac.uk/help-and-support/student-funding}{Student
Funding}

\href{https://studenthub.city.ac.uk/help-and-support/student-health-wellbeing}{Student
Health \& Wellbeing}

\href{https://studenthub.city.ac.uk/help-and-support/support-within-your-school}{School
Student Welfare teams}

\href{https://studenthub.city.ac.uk/help-and-support/student-support-hub-adviser-team}{Student
Support Hub Advisers}

\href{https://www.city.ac.uk/about/facilities/specialist-facilities/student-support-hub}{The
Student Support Hub}: located in Level 1, Drysdale Building, it's
designed to provide in-person access to a wide range of essential
student support services. The Student Support Hub will operate from
9:00am to 6:30pm during term time.

Along with general enquiries, you can visit the following support
services:

\href{https://studenthub.city.ac.uk/help-and-support/english-for-academic-purposes}{Academic
English support}: academic writing and English language support for
international students and for students whose first language is not
English.

\href{https://studenthub.city.ac.uk/help-and-support/improve-your-study-skills}{Academic
Skills}: runs workshops, webinars and one-to-one support sessions to
help you to become a more effective learner.

\href{https://studenthub.city.ac.uk/help-and-support/accommodation-student-hub}{Accommodation}:
can help with the application process, any queries regarding the rooms,
halls, fees and contracts, support if you are looking for accommodation
during the academic year, viewing halls and much more.

\href{https://studenthub.city.ac.uk/careers-and-employability}{Careers
\& Employability}: encompass careers, employability, and student
development activities, including careers resources and appointments,
skills sessions, employer events, community volunteering, professional
mentoring, the Micro-Placement Programme and the City Buddies scheme.

\href{https://studenthub.city.ac.uk/help-and-support/city-cares}{Uni
Cares}: City's dedicated support network for care-experienced students,
young adult carers, young estranged students, refugees and asylum
seekers.

\href{https://studenthub.city.ac.uk/help-and-support/studying-online/digital-skills-and-support}{Digital
Skills}: supports you to effectively use digital technologies and
incorporate them into your learning, student experience and
employability.

\href{https://city-uk-ett.libguides.com/student}{Student Guides to
Educational Technology}: guidance on the use of educational technology
at City St Georges, such as Moodle, Microsoft Teams, and other learning
tools.

\href{https://studentcommunications.newsweaver.com/1w52bcjsfr/18i0udma1fv/external?a=6&p=4118384&t=27571}{AppsAnywhere}:
a platform where you can access many City St George's applications
remotely.

\href{https://studenthub.city.ac.uk/help-and-support/visa-advice-for-international-students/responsibilities}{International
Student \& Visa Advice}: provides confidential advice and guidance to
applicants and students on study-related visa and immigration issues.

\href{https://studenthub.citystgeorges.ac.uk/student-support-services/it-services}{IT
Service Desk}: is the first point of contact for all IT queries.

\href{https://studenthub.city.ac.uk/help-and-support/student-funding}{Student
Funding}: provides advice on student funding available to support your
studies at City.

\href{https://studenthub.city.ac.uk/help-and-support/student-health-wellbeing}{Student
Health \& Wellbeing}: provides practical, emotional and specialist
disability related support to our students.

\href{https://studenthub.city.ac.uk/help-and-support/student-support-hub-adviser-team}{Student
Support Hub Advisers}: helps with:

\begin{itemize}
\tightlist
\item
  Production of ID cards - expired, lost/stolen, new (outside of the
  main registration period)
\item
  Production of a range of letters, student status, bank letter,
  Schengen visa letter
\item
  Help with 18+ oyster card application
\item
  Signing of railcard documents
\item
  Registration (outside of the main registration period)
\item
  Updating student details
\item
  Collection point for graduation certificates
\end{itemize}

\href{https://studenthub.city.ac.uk/help-and-support/student-health-wellbeing/togetherall}{Togetherall}:
it's a safe, online peer support community where you can get and give
support to improve your mental health and wellbeing. The community is
moderated 24/7 by mental health professionals ensuring the safety and
anonymity of all members.

\href{https://reportandsupport.city.ac.uk/}{The Report+Support
platform}: provides staff and students with a confidential way to report
instances of unacceptable behaviour, such as bullying, harassment,
discrimination, hate incidents, domestic abuse, or sexual violence.
Whether you have personally experienced these behaviours or have
witnessed them happening to someone else, the platform offers a
straightforward and accessible way to report your concerns.

\href{https://studenthub.city.ac.uk/help-and-support/stay-city-safe}{Stay
City Safe}: this initiative provides you with essential information on
what to do in case of an incident, how to report it, and where to seek
assistance.

\href{https://studenthub.city.ac.uk/__data/assets/pdf_file/0006/814416/3748-Stay-City-Safe-booklet-AL1i-digital-1.pdf}{Your
guide to staying safe in London}: download our safety booklet.

\section{Students Union}\label{students-union}

\href{https://www.csgsu.co.uk/}{Student's Union}:it's a membership-led
organisation, independent of the University, which exists to make sure
you have the best overall experience at City both socially and
academically. The Union is in Level 2 of the Student Support Hub in
Drysdale Building.

SU offers:

\href{https://csgsu.co.uk/advice-and-support}{Union advice}: on academic
appeals, academic misconduct, complaints, extenuating circumstances and
more.

\href{https://csgsu.co.uk/explore-and-join/societies}{Student
Societies}: these are groups created, run and lead by students for
students. These can be hobbies, campaigns, cultural, academic, basically
anything you can think of. Any student can join a society. Joining a
society is super easy through our website, just click on any society you
are interested in and purchase the standard membership (even if it is
free)!

\href{https://csgsu.co.uk/explore-and-join/start-or-adopt-a-group}{Start
a new society}: Any student at City St George's can apply to set up a
new student group. Setting up a new student group equips you with
employability skills including; leadership, teamwork, communication and
events planning.

\href{https://csgsu.co.uk/explore-and-join/start-or-adopt-a-group}{Society
idea - application form}!

\href{https://events.csgsu.co.uk/}{SU Events}

\href{https://csgsu.co.uk/sports}{Sports}

\href{https://csgsu.co.uk/advice-and-support/finances}{Short Term
Loans}: Interest free loans of up to £200.

\href{https://csgsu.co.uk/your-voice/programme-and-year-representatives}{Programme
rep training}

\href{https://city.unitu.co.uk/Account/Login}{GetHeard@City}: digital
platform for student voice to ask questions to your Programme Reps,
discuss your experiences with other students, and give feedback to the
Students' Union and University staff.

\href{https://csgsu.co.uk/your-voice/programme-and-year-representatives}{Become
a Programme rep}. You are invited to nominate yourself to become a
Program rep. Programme Reps play a key role in ensuring that all
students have a voice and an opportunity to share feedback about their
learning experience. They help students make as many positive changes as
possible and ultimately, help improve the quality of education at City
St George's.

\href{https://forms.office.com/e/6MEdG2EStU}{Programme Rep nomination
form}: usually by early October.

\href{https://csgsu.co.uk/your-voice/programme-and-year-representatives}{more
info about Programme Reps (csgsu.co.uk)}.

\href{studentrep@city.ac.uk}{Questions about Reps}: or visit SU Welcome
Desk in Student Support Hub.

\section{Careers and Employability}\label{careers-and-employability}

Careers \& Employability team encompasses careers, employability, and
student development activities, including careers resources and
appointments, skills sessions, employer events, community volunteering,
professional mentoring, the Micro-Placement Programme and the City
Buddies scheme. Our resourceful careers team can offer expertise and
support, to assist you in developing your career prospects whilst at
City, and beyond.

\href{https://cityuni.sharepoint.com/teams/CUoL-Employability-Economics/SitePages/LearningTeamHome.aspx}{Employability
Economics}

\href{https://studenthub.city.ac.uk/careers-and-employability}{Careers
\& Employability}

Year 2 students can apply to the SPGA Placement and Exchange Programmes:

\begin{itemize}
\tightlist
\item
  SPGA Placement Programme
\item
  SPGA Exchange Programme
\end{itemize}

\href{https://careershub.city.ac.uk/s/resources}{Careers Hub} has
everything you need to start your career journey. Each section has a
list of key information for that topic and links to documents and
websites on the subject.

\href{https://careershub.city.ac.uk/students/events}{Careers Fairs and
employer events}

\href{https://studenthub.city.ac.uk/careers-and-employability/micro-placements-programme}{Micro-Placements}
are an exciting way to gain professional experience via short summer
placements with a wide range of UK-based employers for year 1 and 2 UG
students. The deadline for application is usually in October.

\href{https://studenthub.citystgeorges.ac.uk/career-development/professional-mentoring-scheme}{Professional
Mentoring Scheme}: it's a six-month programme that pairs students with
industry professionals in order to develop their skills, confidence and
future employability. The deadline for application is 23:59, Sunday,
October 6th, 2024.

\href{https://www.unitemps.com/}{Unitemps}: City St George's in-house
temporary jobs agency which offers part time paid temporary work in and
around the University for students and graduates: You can sign up and
apply for roles on the \href{https://www.unitemps.com/}{Unitemps
website}.

\href{https://studenthub.city.ac.uk/careers-and-employability/student-ambassador-scheme}{Student
Ambassador scheme}: Develop your skills, make friends and earn some
extra money by becoming a Student Ambassador at City St George's!

\href{https://careershub.city.ac.uk/students/appointments/topics?campusId=704}{Book
a Career appointment}: The best way to think about your career is in
conversation with one of our expert career consultants. They can respond
directly to your questions and concerns and help give you direction,
even if you are not sure of exactly what career path you want to follow.

\section{Other useful information}\label{other-useful-information}

\href{https://www.city.ac.uk/about/vision-and-strategy/academic-excellence/education/academic-year-and-term-dates}{Term
Dates}

\href{https://studenthub.city.ac.uk/student-administration/registration}{Registration}
- all new and returning students of City, University of London are
required to register for their course each year. If you have any other
questions regarding registration, please email the
\href{mailto:registration@city.ac.uk}{Registration team} If you think
you may be late completing registration and need more time, please
contact the Course Officer for your course within your Department as
this will need to be considered by your academic Course Director.

\href{https://studenthub.city.ac.uk/student-administration/fees-and-finance}{Fees
and Finance}- financial information for students who are studying at
City St George's, University of London.

\href{https://evision.city.ac.uk/?_gl=1*7eeo8m*_ga*NjM2MTY1Mjk3LjE2NjY3MTIwOTU.*_ga_YLSK0292X4*MTY5NDUxMTg4OS4xNC4xLjE2OTQ1MTIxMjkuNjAuMC4w}{Evision-Get
Official Letters and Pay Fees}. You may need a proof of study letter for
various things such as council tax exemption, registering with a doctors
surgery, or even opening a student bank account. You can produce some of
these documents yourselves by logging into Evision with their student
details.

\href{https://www.citysport.org.uk/}{CitySport} - manages the
high-quality sport facilities at CitySport and deliver a range of
health, fitness, and wellbeing initiatives across the University.
\href{https://studenthub.citystgeorges.ac.uk/student-support-services/faith-and-cultural-support}{Faith,
belief and culture at City St George's}

For International Students:

\href{https://studenthub.city.ac.uk/help-and-support/your-health}{Health
information}

\href{https://studenthub.city.ac.uk/help-and-support/visa-advice-for-international-students}{International
Student \& Visa Advice}: advice and guidance to applicants and students
on study-related visa and immigration issues.

\href{https://studenthub.city.ac.uk/help-and-support}{Help and support
\textbar{} Student Hub \textbar{} City, University of London}: You can
find more information about student services.

\section{Expected Behaviour in the
Classroom}\label{expected-behaviour-in-the-classroom}

\subsection{General conduct}\label{general-conduct}

Behave professionally -- be punctual, engaged and nondisruptive in all
teaching and learning sessions.

Respect the physical and digital environment of the institution and our
local community.

Behave responsibly and respectfully.

Be aware that people come from different backgrounds and cultures.

Treat fellow students, staff and visitors with respect.

Behave collaboratively and collegially in shared spaces -- classrooms,
library, computer rooms and online spaces.

\subsection{Attendance}\label{attendance}

It is in your interests to attend; you will gain far more from your
university experience if you attend classes and engage with your
studies, your tutors and your classmates. There is clear evidence that
those who attend classes get better results.

It is also worth noting that your attendance / engagement record may be
requested as part of a reference.

Attendance at classes is monitored; both so that we can comply with UKVI
requirements, for students on visas, and to help us identify students
who may need support.

We won't ask you to explain every absence but we will contact you if
your attendance / engagement raises concerns about your wellbeing.
Remember to take your student card to class so that you can `tap in'.

Do not ``tap into'' a class and then leave before it begins or not long
into it, just to get your attendance recorded. This disturbs classes and
does not benefit you in any way. It is also dishonest.

We conduct manual checks of electronic attendance records and, if you
are found to be falsifying your record in this way, you could be subject
to a disciplinary investigation.

If you can't stay for the whole class, leave during the break, rather
than disturbing others.

Remember all lectures are recorded so, if you can't attend, you can
watch it at a more convenient time.

If you cannot attend a small group class for some good reason you should
request to attend an alternative class, where this is available. Contact
your lecturer or course office for more information.

Arrive on time for all your scheduled classes. Be present and in a seat,
ready to begin the class at the scheduled start time. If there is a
break, return promptly.

\textbf{If you are late:} enter as quietly as possible, and take the
closest available seat so that you minimise the disturbance to others.

In a lecture theatre, use the rear entrance, where available.

\section{Classroom engagement}\label{classroom-engagement}

Prepare the work set.

Whether this is reading and / or carrying out specific exercises, if you
have not prepared you will not get the full benefit from the class.

Being prepared benefits your group; for example, ensuring there is
active discussion; or that you all play your part in the skills based
classes.

If you cannot prepare, for some good reason, you should let your tutor
know. This may prevent you from being put on the spot when called upon
to answer a question which you do not know the answer to.

Engage with the class by:

\begin{itemize}
\item
  Taking part in any polls or quizzes / questions set by the tutor.
\item
  Taking an active part in the smaller group classes. Join in with group
  discussions; answer the questions asked of you; listen and pay
  attention.
\end{itemize}

If you were not able to prepare fully, or at all, pay close attention;
answer or discuss any aspect you are able to. Often you will find you
can take a more active part than you thought.

Don't talk during class, unless you are participating in the class. This
is a distraction to other students who want to listen to the class and
gain the benefit from it. You will not learn from the class if you're
talking about something else.

\subsection{Email communication}\label{email-communication}

Email is our primary means of communication with you; make sure you
check your City St George's email several times a day, especially first
thing in the morning, in case there are any last minute changes to your
timetable due to staff illness.

Remember that email is a formal means of communication: Address all
members of our community politely and formally. When emailing City St
Georges staff, please include your student ID, course name and subject
of query, this helps us to find you on the system and resolve your query
quickly.

Do not say something in an email or message that you would not say to
that person's face, or in a way that you would not want to be addressed
yourself.

Remember that humour or sarcasm can easily be misunderstood, especially
when not accompanied by facial expressions or body language.

Remember that most staff work 9-5 Monday-Friday so you are should not
expect to receive a response outside these times.




\end{document}
